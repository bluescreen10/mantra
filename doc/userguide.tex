% Created 2010-10-11 Mon 16:15
\documentclass[11pt]{article}
\usepackage[utf8]{inputenc}
\usepackage[T1]{fontenc}
\usepackage{graphicx}
\usepackage{longtable}
\usepackage{float}
\usepackage{wrapfig}
\usepackage{soul}
\usepackage{amssymb}
\usepackage{hyperref}


\title{Mantra User Guide}
\author{Mariano Wahlmann}
\date{11 October 2010}

\begin{document}

\maketitle

\setcounter{tocdepth}{3}
\tableofcontents
\vspace*{1cm}

\section{Introduction}
\label{sec-1}

Mantra is a multi-platform General Purporse Language inspired from several languages but mainly in Smalltalk and Perl. Mantra is a Prototype-oriented language.
Typically Object-oriented languages are built using Classes and Objects, in this paradigm objects are nothing but data structures that hold \emph{state} while classes hold the behavior. In languages like Smalltalk classes are also objects - instances of \emph{Class} object.

\section{Basic Syntax}
\label{sec-2}

\subsection{Hello World}
\label{sec-2.1}

Here is an example of ``Hello World'' program in Mantra.
\\
\underline{HelloWorld_\}.__ma\_{}

\begin{verbatim}
#!mantra

System out: 'Hello World!\n'.
\end{verbatim}


\\
This simple program sends the \textbf{out:} message to the global object \textbf{System} with the \textbf{Hello World!\n} string as parameter, and that prints \emph{Hello World!} in the console.
\subsection{Variable Scope}
\label{sec-2.2}


\end{document}
